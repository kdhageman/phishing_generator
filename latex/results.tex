\section{Results}

\subsection{Training the network}
In order to train the network, a dataset of 3,000 phishing domains was used. 
The domains were extracted from a set of phishing URLs\footnote{https://ecrimex.net/}, from which a set of the most popular domain names\footnote{https://www.alexa.com/topsites} was subtracted (e.g. {\tt docs.google.com}).
This subtraction of popular domains was done based on the assumption that popular domain names are not solely registered for phishing purposes, but are rather benign domain being abused for phishing purposes.

All domains were tokenized, transforming them from a string of characters to a one dimensional tensor of integers, where each integer represent a single character. 
The translation between characters and integers was stored in the alphabet of the model.
Before tokenizing, a start symbol {\tt <s>} and a end symbol {\tt </s>} were added to the domain in the start and end respectively. 
The start symbol was needed in order to eventually generate domain names without the need of providing a start symbol manually.
The end symbol was introduced for a similar reason, namely to the network to predict the end of a domain name.

For domain $d$ in the training set, the model was trained as follows:
\begin{enumerate}
\item Initialize the hidden vector to zeroes.
\item For all characters $c_i$ in $d$ (except the last end symbol {\tt </s>}), take its subsequent character $c_{i+1}$ as the training target.
For instance, in the string {\tt ABC}, the target for {\tt A} would be {\tt B}.
\item Predict the output for $c_i$ and compute the loss between the output and the target $c_{i+1}$.
\item Back propagate the (clipped) total loss for domain $d$ through the network.
\end{enumerate}

\subsection{Hyper parameters}
The Adam algorithm was used for the optimization.
For computation of the loss, the negative log likelihood function was used. 
The default values for hyper parameters was used for the majority of the hyper parameters in the different components, with two exceptions. 

First, the size of the hidden layer does not have a default value. 
Secondly, different values for the learning rate of the Adam optimizer were considered.
Figure TODO shows the result of the training phase, for different combinations of parameters.

%\begin{figure}[t]
%	\label{fig:parameter_optimization}
%	\centering
%	\includegraphics[width=0.75\textwidth]{figures/parameter_optimization.eps}
%	\caption{The design of the neural network. The grey boxes are input- and output vectors, the blue boxes represent linear transformation layers and the green boxes are vector transformations applied to the output of the network.}
%\end{figure}

\subsection{Generating domains}
In order to start generating a new phishing domain name, the one-hot encoding of a single character and zero-initialized hidden vector is provided as input to the network.
Then, the network generates a probability distribution for the next character based on the inputs.
From this probability distribution, the character associated with the largest probability is appended to the generated domain name, and its one-hot encoding (in addition to last step's hidden output) is used as the input for the next step.
The generation algorithm runs until the network generates a {\tt </s>} symbol, or when the domain reaches a length of 255 characters (i.e. the maximum size of a domain name).

The dropout node in the model's design ensures that sampling from the model with identical inputs will result in different outputs. 
Therefore, it is possible to use {\tt <s>} as the start symbol for the generation of domains. 
However, because we used a dropout probability $p = 0.1$, the model tends to generate similar domain names.
Therefore, the presented results were generated by choosing a starting character uniformly from the alphabet.

The following shows a sample of generated domains:
\begin{multicols}{3}
\begin{verbatim}
kenterestite.com
89.cem
zanterines.com
ypalecom.com
89.com
lonterestrin.com
inforestine.com.com
erid.cor
veridinetsertere.com
herid.cut
om.com
lonterestine.com.account-infore.com.com
tinetin.com
5.com
2nforestinets.com
\end{verbatim}
\end{multicols}