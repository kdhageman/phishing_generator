\section{Introduction}
Phishing attacks pose a major threat to everyday Internet users, in which victims are persuaded to disclose personal or sensitive information through digital means.
Possibly the most well-known variant is the fake email sent from someone, who claims to be working at a bank or another trustworthy party, asking to visit a website to enter some information.
These phishing emails attempt to persuade their victims to visit a website that looks and feels similar to the one of the targeted brand, but is actually under control of the attacker.

Phishers (i.e. the attackers) put effort into improving the perceived trustworthiness of their communication channels, such as copying email structures, logo's and website layouts.
One of these techniques is to host their fake websites on domains that resemble legitimate ones.
For instance, instead of persuading a victim into visiting the legitimate {\tt paypal.com}, a phisher may redirect the user to {\tt paypal-secure.com} or {\tt paypai.com} instead.
These domains being registered solely for executing phishing attacks are referred to as {\it phishing domains}.

There is a benefit for the general Internet community to identify these domains and to take countermeasures as soon as possible.
In this project, a recurrent neural network is implemented, with the goal of being to generate phishing domains.
The project is a attempt to demonstrate of how a recurrent neural network can capture the specific characteristics of domains in general and phishing domains specifically. 

\subsection{Covered course material}
The primary course material that inspired this project was the introduction to PyTorch by Adam Paszke.
My previous experience with machine learning tools was primarily in {\tt MATLAB} and {\tt scikit-learn}, motivating me to use PyTorch for this project.
The developed neural network in this project is influenced by both the RNN demonstration in the course, as well as one of the PyTorch tutorials\footnote{\url{https://pytorch.org/tutorials/intermediate/char_rnn_generation_tutorial.html}}.