\section{Introduction}
This report describes the development of a recurrent neural network for the generation of phishing domain names.

\subsection{Covered course material}
The primary course material that inspired the project was the introduction to PyTorch by Adam Paszke.
Experience with tools in other machine learning course was primarily with {\tt MATLAB} and {\tt scikit-learn}, motivating me to use PyTorch for this project.

The demonstration of a recurrent neural network for the generation of sentences based 

\subsection{Phishing domain names}
In phishing attacks, an attacker - or phisher - attempts to obtain sensitive information from its victim through digital means.
A common phishing attacks begins with an attacker contacting its victim (e.g. via an email) urging them to visit a particular website (e.g. a spoofed version of a banking website), 
The spoofed website is constructed to persuade the victim in providing sensitive information.

For a phishing attack to succeed, the impersonation of a trusted party should be sufficiently convincing for the victim to believe they are interacting with the actual impersonated party.
Therefore, phishers employ a range of techniques to increase their perceived trustworthiness.

The aforementioned spoofed website is hosted on a particular domain name (e.g. {\tt www.example.org}.
A common technique used by phishers is to host phishing websites on domain names that are similar 