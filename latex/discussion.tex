\section{Discussion}

\subsection{Training}
The impact of learning rate ended up having little effect, because the model seems to converge rather quickly, even with a small learning rate.
Similarly, the size of the hidden layer had little impact on the training loss.

\subsection{Domain generation}
By observing a set of generated domains, there are several interesting observations.

\paragraph{Number as starting character}
Whenever a number is used as the starting number, the generated domain tends to be either (1) very short (e.g. {\tt 5.com} or (2) follow an IP address-like structure (e.g. {\tt 9.19.152.283-328-538-252-558-420-179.49.278.283}).
In the previous example, an IP address ({\tt 9.19.152.283}) seems to get mixed with strings of dash ("-") delimited integers ({\tt 328-538-252-558}).
An interesting observation is that the IP address-like domains contain invalid IP addresses (a single integer within the IP address cannot exceed 255).
Generally speaking, it seems that the model has learned itself to not mix integers with characters.

\paragraph{Domain name structure}
A domain name is composed of several {\it labels}, which represents a hierarchical structure within the domain name.
The labels are delimited by a dot ({\tt .}), where the labels (when read right to left) become more detailed. 
For instance, the right-most label {\tt dk} in {\tt www.example.dk} describes that the domain resides within the Danish domain space, whereas the left label {\tt www} denotes the very specific notion of the domain hosting a web server.
The right-most label in a domain name is referred to as the top-level domain.

The vast majority of the generated domains follows a common domain name structure of having either two or three labels (excluding the aforementioned IP address-like domains).
Interestingly, almost all of these domains have the {\tt .com} (or very close related) top-level domain.
Out of all domains in the training set that have a traditional domain name structure (i.e. ending with valid top level domain), just 56.7\% have the {\tt .com} top level domain. 
Additionally, the {\tt com} label seems to appear in generated domains names not only at the end of a domain name, but also in other places.
For instance, consider the domain {\tt lonterestine.com.account-infore.com.com}, which contains the string {\tt .com} three times.
Concluding, in the generated set of domains, the {\tt .com} top-level domain seems to be overrepresented.

\paragraph{The PayPal syndrome}
As an online payment service, PayPal is one of the most targeted brands when it comes to phishing attacks.
We can see this representation, since many generated domains (in particular those that start with {\tt p}) contain the string {\tt paypal}.

\subsection{Conclusion}
The project proved to be an interesting introduction to recurrent neural networks and gave some interesting insights in the patterns that a recurrent neural network learns.
There are many improvements to be made, most notably the introduction of more 