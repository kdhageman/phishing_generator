\section{Discussion}
By manually observing a set of generated domains, we can observe several interesting patterns in the domains names that the RNN tends to generate.

\paragraph{Domain name structure}
A domain name is composed of several {\it labels}, which represents a hierarchical structure within the domain name.
The labels are delimited by a dot ({\tt .}), where the labels (when read right to left) become more detailed. 
For instance, the right-most label {\tt dk} in {\tt www.example.dk} describes that the domain resides within the Danish domain space, whereas the left label {\tt www} denotes the very specific notion of the domain hosting a web server.
The right-most label (and in cases such as {\tt co.uk} multiple labels) in a domain name is referred to as the top-level domain (TLD).
The label next to the TLD is generally referred to as the second level domain (or 2LD), whereas the remaining set of labels combined is the subdomain of the domain.

Table \ref{tab:statistics} shows an overview of several statistical measures of $T$ and $G$. 
The RNN tends to generate shorter domains than the training set (average length of 20.5 compared to 26.0), although the number of labels is on average almost equal.
The standard deviation in both measures is significantly lower though, suggesting that the generated domains tend to lie closer to each other in terms of domain length and label count.

Domain names can only be registered under standardized TLDs (see the public suffix list\footnote{https://publicsuffix.org/}).
In the training set, 95.2\% of domains contained a valid TLD, where the majority of the 4.9\% invalid domains consists of IP addresses (3.95\% of all training samples).
The generated domains however, only consist of 58\% valid domains. 
Manual inspection of these invalid domain indicates that the generated domains range from almost valid ones (e.g. {\tt iom} or {\tt cou}), to completely bogus ones (e.g. {\tt pverelinybneebioln} or {\tt llaperocre}).

\begin{table}[t]
\centering
\caption{Summary of the training- and generated dataset, $T$ and $G$}
\label{tab:statistics}
\begin{tabular}{@{}llp{1cm}p{1cm}@{}}
\toprule
              &                         & $T$  & $G$  \\ \midrule
\multirow{2}{*}{Domain length} & $\mu$                   & 26.0 & 20.5 \\ 
              & $\sigma$ & 20.5 & 16.3 \\ \midrule
\multirow{2}{*}{Label count}   & $\mu$                   & 3.4  & 3.3  \\ 
              & $\sigma$                & 2.2  & 1.6  \\ \midrule
Valid TLDs    & \%                      & 95.2 & 58.0 \\ \bottomrule
\end{tabular}
\end{table}

\paragraph{IP addresses}
As mentioned previously, the training set contains several domains in the form of IP addresses (e.g. {\tt 111.222.333.443}).
The set of generated domains contains 0.31\% of domains that can be considered to fall in the same category (i.e. having a number as its first character).
A manual inspection seems to indicate that the RNN has not learned from the training set to properly generate an IP address.
Some generated domains contain only two integer (e.g. {\tt 72.171}), whereas others also contain letters (e.g. {\tt 1pp1}).

\subsection{Conclusion}
The domain generator has learned the basics of domain generation: generate domains with a length and label count that resembles actual domains.
However, notable restrictions on domain name were not captured, such as the restriction on TLDs and the structure of IP addresses.
One interesting approach for future work would to introduce Long short-term memory (LSTM) units to the network, since these units tend to be better at capturing information in much older time steps rather than recent steps.